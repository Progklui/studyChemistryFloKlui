\section{Zustandsgrößen und totale Differentiale}

  \subsection{Extensive und Intensive Zustandsgrößen}

    Extensive Zustandsgrößen hängen von der Größe des Systems ab. Intensive   Zustandsgrößen sind demgegenüber unabhängig von der Systemgröße.

      \begin{table}[H]
        \centering
        \caption[Beispiele für Zustandsgrößen, Quelle: Autor]{Beispiele für Zustandsgrößen und Einteilung in intensive und extensive}
        \label{tab:Zustande}
        
        \begin{tabular}{@{}l|l@{}}
          \toprule
            Intensive Zustandsgrößen & Extensive Zustandsgrößen  \\ \midrule
            Molvolumen & Entropie \\
            Temperatur & Stoffmenge \\
            Druck & Volumen \\
            Partialdruck & innere Energie \\
            molare Masse & Enthalpie \\
            Konzentration & Masse \\
            Dichte &  \\
            spezifisches Volumen & \\ \bottomrule
        \end{tabular}
      \end{table}

  \subsection{Das Volumen eines Zylinders ein totales Differential?}
  
    Das Volumen eines Zylinders kann mit
      
      \begin{equation}
        V(r,h) = r^2 \pi h \label{eq:VolumenZylinder}
      \end{equation}            
    berechnet werden. Das totale Differential berechnet sich zu
    
      \begin{equation}
        \text{d}V = \left(\frac{\partial V}{\partial r}\right)_{h} \text{d}r + \left(\frac{\partial V}{\partial h}\right)_{r} \text{d} h = 2r\pi h \text{d}r + r^2 \pi \text{d} h. \label{eq:totalesDifferentialAllgemein} 
      \end{equation}
    Um zu überprüfen, ob $V(r,h)$ eine Zustandsfunktion ist, muss untersucht werden, ob die gemischten zweiten partiellen Ableitungen gleich sind. Auf einem sternförmigen Gebiet gelten dann die Integrabilitätsbedingungen sowie der Satz von Schwarz. Es gilt 
    
      \begin{equation}
        \begin{split}
          \partial_{r}\partial_{h} V &= 2 r\pi  \\
          \partial_{h}\partial_{r} V &= 2 r\pi. \label{eq:ZeiteAbleitungVolumen}
        \end{split}
      \end{equation}
    Daraus folgt, dass $\partial_{r}\partial_{h} V = \partial_{h}\partial_{r} V$ und damit ist $V(r,h)$ eine Zustandsfunktion. $\square$

  \pagebreak
  
  \subsection{Totales Differential einer exemplarischen Funktion}
  
    Wir betrachten die Funktion
    
      \begin{equation}
        f(x,y) = x^4 + xy
      \end{equation}
    und bilden $J = \begin{pmatrix} \partial_{x} \, f(x,y) & \partial_y \, f(x,y)\end{pmatrix} = \begin{pmatrix} 4x^3 + y & x \end{pmatrix}$
    sowie die Hesse-Matrix 
    
      \begin{equation} 
        H_f = \begin{pmatrix} \partial_{xx} \, f(x,y) & \partial_{yx} \, f(x,y)\\ \partial_{xy} \, f(x,y)& \partial_{yy} \, f(x,y) \end{pmatrix} = \begin{pmatrix} 12x^2& 1\\ 1& 0 \end{pmatrix}.
      \end{equation}
    Aufgrund der Symmetrie von $H_f$ ist auf einem sternförmigen Gebiet der Satz von Schwarz erfüllt und damit $f(x,y)$ eine Zustandsfunktion. Das totale Differential lässt sich schreiben als
    
      \begin{equation}
        \text{d} f(x,y) = \left(\partial_{x} \, f(x,y)\right)_y \text{d} x + \left(\partial_{y} \, f(x,y)\right)_x \text{d} y = \left(4x^3 + y\right) \, \text{d} x + \left(x\right) \, \text{d} x,
      \end{equation}
    wobei die Einträge der oben angegebenen Jakobi-Matrix eingesetzt wurden.
    
  \subsection{Das molare Volumen ein totales Differential?}
  
    Wir betrachten das totale Differential
    
      \begin{equation}
        \text{d} V_m = \left(\frac{R}{p}\right)_p \text{d} T - \left(\frac{RT}{p^2}\right)_T \text{d}p.
      \end{equation}
    Die gemischten zweiten partiellen Ableitungen lauten
    
      \begin{equation}
        \begin{split}
          \partial_{p}\partial_{T} \, V_m &= -\frac{R}{p^2} \\
          \partial_{T}\partial_{p} \, V_m &= -\frac{R}{p^2}
        \end{split}
      \end{equation}
    und damit ist $V_m$ eine Zustandsfunktion, was wiederum impliziert, dass $\text{d} V_m$ wirklich ein totales Differential ist.
    
  \subsection{Zusammensetzung eines Gasgemisches}
  
    Folgende Daten eines Gasgemisches der drei Komponenten A, B, C sind gegeben: $p_{Ges.} = \SI[mode=text]{1.00}{\bar}$, $V_{Ges.} = \SI[mode=text]{1}{m^3}$, $T = \SI[mode=text]{298}{\kelvin}$, $x_A = \SI[mode=text]{0.3}{}$ und $p_B = \SI[mode=text]{0.25}{\bar}$. Wir rechnen wie folgt:
    
    \begin{equation}
      \begin{split}
        p_A &= x_A \cdot p_{Ges.} \\
        \Rightarrow p_C &= p_{Ges.} - p_B - p_A \\
        \Rightarrow n_C &= \frac{p_C \cdot V_{Ges.}}{RT} \\
        \Rightarrow m_C &= n_C \cdot M_{\ch{N2}} = \SI[mode=text]{508.6}{\gram}
      \end{split}
    \end{equation}
    Das Gasgemisch enthält demnach \SI[mode=text]{508.6}{\gram} an \ch{N2}.