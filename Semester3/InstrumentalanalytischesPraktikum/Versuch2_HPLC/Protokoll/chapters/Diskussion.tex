\section{Fazit und Diskussion}

  Die Ergebnisse der qualitativen und quantitativen Analyse sind großteils kohärent. Die qualitative Zusammensetzung des Pharmazeutika-Mix konnte durch Vergleich der Kapazitätsfaktoren gut bestimmt werden. Diese stimmen bis auf Carbamazepin bis zur inkl. zweiten Nachkommastellen exakt überein, was die Robustheit und Zuverlässigkeit der Methode (HPLC) bestätigt. 
  
  Da das Bestimmtheitsmaß der Kalibriergeraden annähernd gleich 1 ist und keine offensichtlichen Ausreißer vorhanden sind, kann davon ausgegangen werden, dass die Kalibrierung erfolgreich war. Potentielle Fehlerquellen wären hierbei mehrmaliges falsches Pipettieren, was aber in diesem Fall ausgeschlossen werden kann. Falsches Pipettieren bei einer Kalibrierlösung könnte man zum Beispiel durch einen Ausreißertest nachweisen. Grundsätzliche systematische Fehler sind natürlich nicht auszuschließen, man hätte aber in diesem Fall auch keine Möglichkeiten, diese aufzufinden. Demenstprechend kann die Analyse im Rahmen des Möglichen als erfolgreich betrachtet werden.  