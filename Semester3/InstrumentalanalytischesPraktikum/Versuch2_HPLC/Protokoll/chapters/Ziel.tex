\section{Ziel des Experiments}
  
  Eine Studie konnte zeigen, dass die Medikamentenrückstände in Süßwasser weltweit deutlich gestiegen sind. Auch das Trinkwasser sei in manchen Gebieten betroffen. Kommen die Medikamente in Kontakt mit Ökosystemen, können sie dort erhebliche Schäden anrichten. Antibiotika beispielsweise zerstören Bakterien, welche für die Stabilität des Ökosystems von großer Bedeutung sind. Auch ist der Weg zur Entwicklung von Antibiotika-Resistenten Bakterien geebnet. \citep{MedicamentsORF} Ursachen für die Rückstände sind unter anderem natürliche Exkretion bzw. nicht fachgerechte Entsorgung. \citep{Versuchsvorschrift} Die Möglichkeit, Medikamentenrückstände in Wasser nachzuweisen, gewinnt deswegen zunehmend an Bedeutung.
  
  Motiviert durch diese Probleme soll in diesem Experiment die Zusammensetzung eines Pharmzeutika-Mix sowohl qualitativ als auch quantiativ bestimmt werden. Dies geschieht unter Verwendung einer Umkehrphasen-HPLC. Auch sollten die Grundlagen der Chromatographie erarbeitet werden. 