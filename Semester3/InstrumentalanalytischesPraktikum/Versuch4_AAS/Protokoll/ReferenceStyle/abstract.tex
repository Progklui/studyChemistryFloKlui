Die Bestimmung der Blei- und Calcium Konzentration einer Probe ist aus biologischen Gründen von großer Bedeutung, da die Über- bzw. Unterschreitung von Grenzwerten zu erheblichen gesundheitlichen Schäden führen kann. Der Blei-Gehalt einer Probe wurde mithilfe von AAS durch Standardaddition bestimmt und beträgt: 

\noindent $c_{\ch{Pb}} = \SI[mode=text, multi-part-units = brackets, separate-uncertainty]{126(27)}{ppm} \left(N = 15, s_c = \pm \SI[mode=text]{12.6}{ppm}, \alpha = 0.05\right)$. Der Calcium-Gehalt einer Probe wurde mithilfe von FES durch externe Kalibrierung bestimmt und beträgt: 

\noindent $c_{\ch{Ca}} = \SI[mode=text, multi-part-units = brackets, separate-uncertainty]{11.3(3)}{ppm} \left(N = 15, s_c = \pm \SI[mode=text]{0.16}{ppm}, \alpha = 0.05\right)$.