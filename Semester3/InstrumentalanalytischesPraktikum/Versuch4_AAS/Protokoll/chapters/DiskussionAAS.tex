\subsection{Fazit und Diskussion}
  
  Wie in Abbildung \ref{fig:KalibriergeradeBlei} zu sehen, weichen mehrere Messpunkte etwas stärker von der Kalibriergeraden ab. Ein Ausreißertest würde sich anbieten, wurde aber nicht durchgeführt, da dies die Statistik durch die geringere Anzahl an Messungen negativ beeinflussen würde. Diese Abweichungen erklären das im Vergleich zur FES von Calcium (siehe Kapitel \ref{sec:ErgebnisseFES}) kleinere Bestimmtheitsmaß und den relativ großen Vertrauensbereich bei $\alpha = 0.05$, welcher in etwa \SI[mode=text]{20}{\percent} der Probenkonzentration entspricht. Die Abweichungen lassen sich aber vermutlich auf einen Fehler des Geräts zurückführen, da bei jeder Messung (bis auf jener bei \SI[mode=text]{20}{ppm}) nur maximal ein Messwert stärker abweicht. Ein Pipettierfehler würde vermutlich eine wesentlich deutlichere Abweichung von der Kalibriergeraden zur Folge haben. 
  
  Der Grenzwert für den \ch{Pb}-Gehalt in Trinkwasser ist seit 1. Dezember 2013 auf \SI[mode=text]{10}{\micro\gram\per\liter} festgelegt. Dies entspricht einer Konzentration \SI[mode=text]{0.01}{\milli\gram\per\liter} bzw. \SI[mode=text]{0.01}{ppm}.\citep{UmweltbundesamtBlei} Von der Bundes-Bodenschutz- und Altlastenverordnung (Deutschland) werden die Prüfwerte für die Schwermetallbelastung des Bodens festgelegt. Für Kinderspielplätze liegt der Prüfwert bei \SI[mode=text]{200}{\milli\gram\per\kilo\gram} und in Wohngebieten bei \SI[mode=text]{400}{\milli\gram\per\kilo\gram}, was  \SI[mode=text]{200}{ppm} bzw. \SI[mode=text]{400}{ppm} entspricht.\citep{BodenBlei} 
  
  Der \ch{Pb}-Gehalt der Probe ist damit um einen Faktor $10^4$ größer wie der Grenzwert in Trinkwasser, liegt jedoch in derselben Größenordnung wie die Prüfwerte für Bodenproben von Kinderspielplätzen bzw. Wohngebieten.