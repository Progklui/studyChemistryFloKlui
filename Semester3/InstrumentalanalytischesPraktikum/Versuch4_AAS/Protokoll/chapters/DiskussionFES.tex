\subsection{Fazit und Diskussion}
  
  Wie in Abbildung \ref{fig:KalibriergeradeEisen} zu sehen, werden die Messpunkte durch die Kalibriergerade gut beschrieben, was durch das hohe Bestimmtheitsmaß (nahe an 1) bestätigt wird. Dementsprechend ist der Vertrauensbereich kleiner wie bei der AAS von Blei (siehe Kapitel \ref{sec:ErgebnisseAAS}). 
  
  Da \ch{Ca} eine große biologische Bedeutung (u. a. als Cofaktor) hat, wird der Grenzwert für den Gehalt in Trinkwasser mit \SI[mode=text]{400}{ppm} im Vergleich zu \ch{Pb} sehr hoch angesetzt.\citep{CalciumGrenzwert} Es ist jedoch zu hinterfragen, ob ein solcher überhaupt Sinn macht, da ernsthafte gesundheitliche Folgen (bis auf die Bildung von Nierensteinen) nicht bekannt sind. Ein Vergleich mit dem Gehalt in Bodenproben wird aus diesem Grund nicht gemacht. 
  
  Der \ch{Ca}-Gehalt der Probe liegt damit deutlich unter diesem (fragwürdigen) Grenzwert in Trinkwasser. 