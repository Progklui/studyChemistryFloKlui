\section{Ziel des Experiments}

  Blei ist eine toxisches Metall, das große gesundheitliche Schäden im Körper anrichten kann. Da früher unter anderem Trinkwasserleitungen aus Blei angefertigt wurden, war die Bleibelastung in der Bevölkerung relativ hoch, weswegen von Regierungen Grenzwerte für die Blei-Konzentration festgesetzt wurden.\citep{UmweltbundesamtBlei} Um die Einhaltung dieser Grenzwerte überprüfen zu können, stehen hochmoderne instrumentelle Geräte und Methoden zur Verfügung. Im Vergleich zu den nasschemischen Methoden sind diese wesentlich empfindlicher mit einer niedrigeren Nachweisgrenze. Ein Beispiel einer solchen Methode ist die Flammen-Atomsspektroskopische Bestimmung von Blei. \citep{Mikromethode} 
  
  Calcium ist im Vergleich zu Blei notwendig für die Gesundheit. Die Gehaltsbestimmung ist sinnvoll, um zum Beispiel die Qualität des Trinkwassers zu überprüfen. Ein zu geringer Gehalt ist z. B. schlecht, da ansonsten häufig die Konzentration an Natrium höher ist, was negative Auswirkungen auf die Gesundheit hat.\citep{CalciumGrenzwert} \\
  
  Das Ziel des Experiments ist also die quantitative Bestimmung des Blei- und Calcium-Gehalts zweier Proben. Der Blei-Gehalt wird mithilfe von Atomabsorptionsspektroskopie durch Standardaddition bestimmt. Der Calcium-Gehalt wird mithilfe von Flammenemissionsspektroskopie durch externe Kalibrierung bestimmt. Weiters sollen die theoretischen Grundlagen aufgearbeitet werden, um ein besseres Verständnis der verwendeten Methoden zu entwickeln. 