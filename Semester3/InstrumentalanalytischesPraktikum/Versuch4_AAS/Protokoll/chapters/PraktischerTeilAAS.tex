\subsection{Praktischer Teil}
  
  In der folgenden Tabelle werden die Geräteparameter sowie die verwendeten Chemikalien aufgelistet.
  
  \begin{table}[H]
    \centering
    \caption[Geräteparameter und Chemikalienliste, Quelle: Autor]{Geräteparameter und Chemikalienliste}
    
    \label{tab:GerateparameterAAS}
    \begin{tabular}{@{}ll|lp{4.5cm}l@{}}
      \toprule
      Geräteparameter & & Materialien   \\ \midrule
        Gerät: &  & Probelösung  \\ 
        Methode: &  & Blei-Urstandardlösung (\SI[mode=text]{1000}{ppm} \\
        Brenngas: &  & Deionat \\
        Oxidans: &  &  \\
        Brenner: &  &  \\
        Lampe: &  &  \\ \bottomrule
    \end{tabular}
  \end{table}
  
  \noindent Die Probelösung wird im \SI[mode=text]{100}{\milli\liter} Messkolben unter homogenisieren bis zur Marke mit Deionat aufgefüllt. In einem \SI[mode=text]{100}{\milli\liter} Messkolben werden \SI[mode=text]{100}{\milli\liter} eines \SI[mode=text]{50}{ppm} hergestellt. Dazu werden \SI[mode=text]{5}{\milli\liter} eines \SI[mode=text]{1000}{ppm} \ch{Pb} Ur-Standards (Einwagge von \SI[mode=text]{1.598}{\gram\per\liter} an \ch{Pb(NO3)2}) mit einer \SI[mode=text]{5}{\milli\liter} Vollpipette entnommen und auf \SI[mode=text]{100}{\milli\liter} verdünnt. Mit der \SI[mode=text]{10}{\milli\liter} Vollpipette werden in 5 \SI[mode=text]{50}{\milli\liter} Messkolben je \SI[mode=text]{10}{\milli\liter} Probelösung pipettiert. Zu diesen Messkolben werden mit der Vollpipette 0, 5, 10, 15 und \SI[mode=text]{20}{\milli\liter} des \SI[mode=text]{50}{ppm} \ch{Pb} Standards zugegeben, bis zur Marke aufgefüllt und geschüttelt. Man erhält 5 Lösungen mit gleichen Proben-, jedoch unterschiedlichen Standardkonzentration. \\
  
  Vor einer Messung wird die Ansaugkapillare in deionisiertes Wasser getaucht und eine Leermessung zur Kontrolle der Reinheit durchgeführt. Anschließend wird die Ansaugkapillare abgewischt und in die Probelösung mit der niedrigsten Konzentration eingetaucht. Nach einer Ansaugzeit von ca. \SI[mode=text]{3}{s} wird die Messung gestartet, wobei das Gerät bei jeder Probelösung die Extinktion dreimal misst und am Bildschirm ausgibt. Die Messwerte werden notiert, die Ansaugkapillare wird abgewischt und ein neuer Messvorgang gestartet.