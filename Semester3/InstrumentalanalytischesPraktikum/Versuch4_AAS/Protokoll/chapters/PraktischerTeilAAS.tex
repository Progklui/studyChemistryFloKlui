\subsection{Praktischer Teil} \label{sec:PraktischerTeilAAS}
  
  In der folgenden Tabelle werden die Geräteparameter sowie die verwendeten Chemikalien aufgelistet.
  
  \begin{table}[H]
    \centering
    \caption[Geräteparameter und Chemikalienliste, Quelle: Autor]{Geräteparameter und Chemikalienliste}
    
    \label{tab:GerateparameterAAS}
    \begin{tabular}{@{}lll|lp{4.5cm}l@{}}
      \toprule
      Geräteparameter &  &  & Materialien   \\ \midrule
        Gerät & AAnalyst 800 PERKIN ELMER &  & Probelösung  \\ 
        Temperatur & 2200-\SI[mode=text]{2500}{\degreeCelsius} &  & Blei-Urstandardlösung (\SI[mode=text]{1000}{ppm}) \\
        Brenngas & Acetylen (Flussrate \SI[mode=text]{2.0}{\milli\liter\per\minute}) &  & Deionat \\
        Oxidans & Luft (Flussrate \SI[mode=text]{17.0}{\milli\liter\per\minute}) &  &  \\
        Wellenlänge & \SI[mode=text]{283.3}{\nano\meter} &  &  \\
        Lampenstrom & \SI[mode=text]{11}{\milli\ampere} &  &  \\ \bottomrule
    \end{tabular}
  \end{table}
  
  \noindent Die Probelösung wird im \SI[mode=text]{100}{\milli\liter} Messkolben unter homogenisieren bis zur Marke mit Deionat aufgefüllt. In einem \SI[mode=text]{100}{\milli\liter} Messkolben werden \SI[mode=text]{100}{\milli\liter} eines \SI[mode=text]{50}{ppm} \ch{Pb} Standards hergestellt. Dazu werden \SI[mode=text]{5}{\milli\liter} eines \SI[mode=text]{1000}{ppm} \ch{Pb} Ur-Standards mit einer \SI[mode=text]{5}{\milli\liter} Vollpipette in einen \SI[mode=text]{100}{\milli\liter} Messkolben überführt und  bis zur Marke aufgefüllt. Mit der \SI[mode=text]{10}{\milli\liter} Vollpipette werden in 5 \SI[mode=text]{50}{\milli\liter} Messkolben je \SI[mode=text]{10}{\milli\liter} Probelösung pipettiert. Zu diesen Messkolben werden mit der Vollpipette 0, 5, 10, 15 und \SI[mode=text]{20}{\milli\liter} des \SI[mode=text]{50}{ppm} \ch{Pb} Standards zugegeben, bis zur Marke aufgefüllt und geschüttelt. Man erhält 5 Lösungen mit gleichen Proben-, jedoch unterschiedlichen Standardkonzentration. \\
  
  Vor einer Messung wird die Ansaugkapillare in deionisiertes Wasser getaucht und eine Leermessung zur Kontrolle der Reinheit durchgeführt. Anschließend wird die Ansaugkapillare abgewischt und in die Probelösung mit der niedrigsten Konzentration eingetaucht. Nach einer Ansaugzeit von ca. \SI[mode=text]{3}{s} wird die Messung gestartet, wobei das Gerät bei jeder Probelösung die Extinktion dreimal misst und am Bildschirm ausgibt. Die Messwerte werden notiert, die Ansaugkapillare wird abgewischt und ein neuer Messvorgang gestartet.
  
    \subsection{Herstellung eines \SI[mode=text]{1000}{ppm} \ch{Pb} Ur-Standards}
    
      Im Folgenden wird die Herstellung von \SI[mode=text]{1}{\liter} eines \SI[mode=text]{1000}{ppm} \ch{Pb} Ur-Standards beschrieben. Die Konzentration an \ch{Pb} sollte also \SI[mode=text]{1000}{\milli\gram\per\liter} betragen. Die benötigte Stoffmenge kann mit 
        
        \begin{equation}
          n_{\ch{Pb}} = \frac{m_{\ch{Pb}}}{M_{\ch{Pb}}} = \frac{\SI[mode=text]{1}{\gram}}{\SI[mode=text]{207.2}{\gram\per\mole}} = \SI[mode=text]{0.00483}{\mole}
        \end{equation}   
      berechnet werden. Daraus kann die Einwaage an \ch{Pb(NO3)2} gemäß 
        
        \begin{equation}
          m_{\ch{Pb(NO3)2}} = n_{\ch{Pb}} M_{\ch{Pb(NO3)2}} = \SI[mode=text]{1.599}{\gram}
        \end{equation}
      berechnet werden. Diese Menge wird in einem \SI[mode=text]{1000}{\milli\liter} Maßkolben eingewogen und bis zur Marke aufgefüllt. 
           