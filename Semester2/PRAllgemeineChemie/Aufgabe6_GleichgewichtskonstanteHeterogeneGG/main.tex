\documentclass{article}
\usepackage[utf8]{inputenc}
\usepackage[english,ngerman]{babel}
%% ========================================================================
%%%% MISC usepackages
%% ========================================================================

%% Chemistry
\usepackage{chemfig,chemmacros}
\chemsetup{modules = all}
\chemsetup[redox]{explicit-sign = true}
\chemsetup[phases]{pos=sub}
%\chemsetup[reactions]{before-tag = {R}, tag-open = [, tag-close = ]}
  
%% Maths
\usepackage{amsmath,amssymb,amsthm,textcomp}

%% Physics
\usepackage{siunitx}

%% Graphics
\usepackage{graphicx}
\usepackage{tikz}
\usepackage{rotating}
%\usepackage{subfig}

%% Tables and Lists
\usepackage{enumerate}
\usepackage{multicol}
\usepackage{geometry}
\usepackage{tabu}
\usepackage{listings}
\usepackage{tabularx}

%% Structures and Style
\usepackage{caption}
\usepackage{subcaption}
\usepackage{booktabs}
\usepackage{colortbl}

\usepackage{xcolor}
\usepackage{xfrac}
\usepackage[export]{adjustbox}[2011/08/13]

\usepackage{booktabs}
\usepackage{float}

%% Citing and Settings
\usepackage[backend=biber,
style=numeric,
backref=true, 
natbib=true, %% offering natbib-compatible commands
hyperref=true, %% using hyperref-package references
sorting= none,
doi=true,
maxcitenames=10,
maxbibnames=100,
citestyle=numeric
]{biblatex}

\addbibresource{references.bib}

\usepackage[toc,automake]{glossaries}
\include{abbrevations}
\makeglossaries

\usepackage[colorlinks=true,linkcolor=blue]{hyperref}

%% Figure settings
\renewcommand{\figurename}{Abbildung}
\renewcommand{\tablename}{Tabelle}
\renewcommand{\listfigurename}{Abbildungsverzeichnis}
\renewcommand{\listtablename}{Tabellenverzeichnis}

%% ========================================================================
%%%% Document Information
%% ========================================================================

%% Title
\title{Bestimmung der Gleichgewichtskonstante für ein heterogenes Gleichgewicht} % Title
\author{Autor: Florian \textsc{Kluibenschedl}} % Author name
\date{Bericht verfasst am: \today} % Date for the report

\begin{document}
  \renewtagform{reaction}[Rgl. ]{}{}
  
  \maketitle % Insert the title, author and date
  
  \begin{center}
    \begin{tabular}{r l}
      Versuchsdurchführung am: & 04. März 2019\\ % Date the experiment was performed
      Gruppe, Matrikelnummer: & 3, 11805747 \\
      Assistent: & Professor Smith % Instructor/supervisor
    \end{tabular}
  \end{center}


  \begin{abstract}
    
  \end{abstract}
  
  \section{Theoretische Grundlagen}
  
    \subsection{Motivation} \label{sec:Motivation}
    
      Salze besitzen in Wasser unterschiedliche Löslichkeiten. Kochsalz \ch{NaCl} löst sich beispielsweise sehr gut. Beim gelben \ch{PbI2} handelt es sich um ein schwerlösliches Salz. Die entsprechende Lösereaktion wird in \ref{rec:LosungPblei} beschrieben. 
           
      \begin{reaction}
        PbI2\sld{} <<=> Pb\pch[2]\aq{} +  2 * I\mch\aq{} \label{rec:LosungPblei}
      \end{reaction}
      
      Werden nun zu einer Lösung von \ch{Pb\pch[2]\aq} Ionen \ch{I\mch\aq} Ionen hinzugegeben, fällt \ch{PbI2\sld} aus, wenn das Löslichkeitsprodukt knapp überschritten wird. Dieser Punkt ist erkennbar durch den charakteristischen, gelben Niederschlag von \ch{PbI2\sld}. Es liegt also eine annähernd gesättigte Lösung vor und man kann annehmen, dass das Ionenprodukt von \ch{Pb\pch[2]\aq} und \ch{I\mch} dem Löslichkeitsprodukt entspricht. Ist die Menge an beteiligtem \ch{Pb\pch[2]\aq} und \ch{I\mch\aq} bekannt, kann somit über das Ionenprodukt das Löslichkeitsprodukt berechnet werden.
      
      Da Blei unter anderem giftig ist, ist eine exakte Kenntnis der Bleikonzentration vonnöten, um z. B. zu überprüfen, ob bestimmte Grenzwerte eingehalten werden. Die Konzentration von \ch{Pb\pch[2]\aq} Ionen kann bei bekanntem Löslichkeitsprodukt berechnet werden.
   
    \subsection{Ziel des Experiments}
    
      Auf Basis der obigen Überlegungen ist das Ziel, über Titration eine möglichst exakte Bestimmung des Löslichkeitsproduktes von \ch{PbI2\sld} durchzuführen.
    
  \section{Experimenteller Teil}
  
    \subsection{Verwendete Materialien}
              
      \begin{table}[H]
        \centering
        \caption[Materialienliste, Quelle: Autor]{Auflistung der verwendeten Geräte und Chemikalien}
        \label{tab:Materialien}
        
        \begin{tabular}{@{}ll|ll@{}}
          \toprule
            Geräte & Hersteller & Chemikalie & Hersteller \\ \midrule
            \SI[mode=text]{600}{\milli\liter} Becherglas &  & \SI[mode=text]{0.5}{M} \ch{Pb(NO3)2} Lösung &  \\
            \SI[mode=text,separate-uncertainty=true]{25.000(75)}{\milli\liter} Bürette &  & \SI[mode=text]{0.02}{M} \ch{KI} Lösung & \\
            \SI[mode=text,separate-uncertainty]{25.000(45)}{\milli\liter} Vollpipette &  & deionisiertes \ch{H2O} &  \\
            \SI[mode=text,separate-uncertainty]{100}{\milli\liter} Erlenmeyerkolben &  &  &  \\
            digitales Thermometer &  &  &  \\
            Stativ mit Klammern &  &  &  \\
            Magnetrührer &  &  &  \\
            Rührfisch &  &  &  \\ \bottomrule
        \end{tabular}
      \end{table}
    
    \subsection{Versuchsdurchführung}  \label{sec:Versuch}
    
      Zunächst wurden \SI[mode=text]{25}{\milli\liter} einer \SI[mode=text]{0.5}{M} \ch{Pb(NO3)3} Lösung in einen \SI[mode=text]{100}{\milli\liter} Erlenmeyerkolben pipettiert und mit einer \SI[mode=text]{0.02}{M} \ch{KI} Lösung bis zum Endpunkt titriert. Dabei wurde die Lösung konstant mit dem Rührfisch und Magnetrührer bei \SI[mode=text]{400}{rpm} gerührt. Am Endpunkt konnte das ausfallende \ch{PbI2\sld} (gelbe) nicht mehr durch Rühren gelöst werden.
      
      Die nächsten Titrationen erfolgten bei unterschiedlichen Konzentrationen der \ch{Pb(NO3)3} Lösung. Dazu wurde eine Verdünnungsreihe erstellt. \SI[mode=text]{25}{\milli\liter} der \SI[mode=text]{0.5}{M} \ch{Pb(NO3)3} Lösung wurden mit \SI[mode=text]{25}{\milli\liter} deionisiertem Wasser verdünnt. Von der resultierenden \SI[mode=text]{0.25}{M} \ch{Pb(NO3)3} Lösung wurden \SI[mode=text]{25}{\milli\liter} mit der \SI[mode=text]{0.02}{M} \ch{KI} Lösung wie oben bescchrieben titriert. Der verbleibende Rest der verdünnten Lösung (\SI[mode=text]{25}{\milli\liter}) wurde mit deionisiertem Wasser auf \SI[mode=text]{50}{\milli\liter} aufgefüllt. Die resultierende \SI[mode=text]{0.125}{M} \ch{Pb(NO3)3} Lösung wurde wie oben beschrieben titriert und verdünnt. Diese Prozedur wurde wiederholt und man erhielt eine \SI[mode=text]{0.063}{M}, \SI[mode=text]{0.031}{M} und \SI[mode=text]{0.016}{M} \ch{Pb(NO3)3} Lösung. Der Verbrauch an \ch{KI} Lösung sowie die Temperatur am Endpunkt der Titration\footnote{es wurde darauf geachtet, starke Temperaturunterschiede zu vermeiden, da die Löslichkeit stark temperaturabhängig ist und die Ergebnisse sonst nicht vergleichbar wären} wurden jeweils notiert. Die gesättigte \ch{PbI2} Lösung wurde im Schwermetall-Abfall entsorgt.
    
     
    \subsection{Auswertung} \label{sec:Auswertung}
      
      Die Konzentration von \ch{Pb\pch[2]\aq} und \ch{I\mch\aq} kann mit dem von der Gesamtkonzentration abhängigen Löslichkeitsprodukt $K_{L,c}$ berechnet werden - siehe \eqref{eq:Klc}. Dabei wird angenommen, dass die Aktivität des Feststoffes \ch{PbI2\sld} gleich eins ist. \\
      
      \begin{equation}
        K_{L,c} = \frac{a_{\ch{Pb\pch[2]\aq}} * a_{\ch{I\mch\aq}}^2}{a_{\ch{PbI2\sld}}} = a_{\ch{Pb\pch[2]\aq}} * a_{\ch{I\mch\aq}}^2 = f_{\ch{Pb\pch[2]\aq}} [\ch{Pb\pch[2]\aq}] * f_{\ch{I\mch\aq}}^2 [\ch{I\mch\aq}]^2 \label{eq:Klc}
      \end{equation} \\
      
      Unter der Annahme, dass die Wechselwirkungen zwischen den Ionen und Molekülen der Lösung vernachlässigt werden können, ergibt sich aus \eqref{eq:Klc} das bekannte Löslichkeitsprodukt $K_{L}$ - siehe \eqref{eq:Kl}. Es gilt also $\lim_{c_{ges.}\to 0} f_{i} = 1 \Leftrightarrow \lim_{c_{ges.}\to 0} K_{L,c} = K_{L}$.
      
      \begin{equation}
        K_{L} = [\ch{Pb\pch[2]\aq}] * [\ch{I\mch\aq}]^2 \label{eq:Kl}
      \end{equation} \\
      
      Die symbolische Konzentration von \ch{PbI2} in der Lösung wird durch die Löslichkeit $L_{M}$ ausgedrückt. Sie errechnet sich aus dem Löslichkeitsprodukt - siehe \eqref{eq:solubility}.
      
      \begin{equation}
        L_{M} = [\ch{Pb\pch[2]\aq}] = \frac{1}{2} [\ch{I\mch\aq}] = \sqrt[3]{\frac{K_{L}}{4}} \label{eq:solubility}
      \end{equation} \\
      
      Im Folgenden werden weitere Beziehungen angegeben, die für die Berechnung der Werte in Tabelle \ref{tab:Messdaten} benötigt wurden. Die Berechnung von [\ch{Pb(NO3)3}] wird nicht angeführt, da dies bereits in \ref{sec:Versuch} erledigt wurde.
      
      \begin{equation}
        pK_{L,c} = -\log_{10} (K_{L,c})
      \end{equation}
      
      \begin{equation}
        V_{Ende} = V_{\ch{KI}} + V_{\ch{Pb(NO3)3}}
      \end{equation}
      
      \begin{equation}
        [\ch{Pb\pch[2]\aq}]_{eq.} = \frac{[\ch{Pb(NO3)3}] * V_{\ch{Pb(NO3)3}}}{V_{Ende}}
      \end{equation}
      
      \begin{equation}
        [\ch{I\mch\aq}]_{eq.} = \frac{[\ch{KI}] * V_{\ch{KI}}}{V_{Ende}}
      \end{equation}
      
      
      
    \subsection{Messergebnisse und Literaturwerte}
    
      In Tabelle \ref{tab:Messdaten} sind alle Messwerte, die im Rahmen der Versuchsdurchführung wie in \ref{sec:Versuch} beschrieben, gemessen wurden. Volumina sind in ml, Konzentrationen in M angegeben. Die einzelnen aus den Messdaten errechneten Werte wurden gemäß den in \ref{sec:Auswertung} angeführten Gleichungen berechnet.
      
      \begin{table}[H]
        \centering
        \caption[Messdaten und daraus abgeleitete Größen, Quelle: Autor]{Messdaten und daraus abgeleitete Größen}
        \label{tab:Messdaten}
          \begin{tabular}{@{}l|llll|ll|lll@{}}
            \toprule
             Nr. & $V_{\ch{Pb(NO3)3}}$ & [\ch{Pb(NO3)3}] & $V_{\ch{KI}}$ & $V_{Ende}$ & $[\ch{Pb\pch[2]\aq}]_{eq.}$ & $[\ch{I\mch\aq}]_{eq.}$ & $K_{L,c}$ & $pK_{L,c}$ & T in \si{\degreeCelsius} \\ \midrule
             1 & 25 & 0.5 &  &  &  &  &  &  \\
             2 & 25 & 0.25 &  &  &  &  &  &  \\
             3 & 25 & 0.125 &  &  &  &  &  &  \\
             4 & 25 & 0.063 &  &  &  &  &  &  \\
             5 & 25 & 0.031 &  &  &  &  &  &  \\
             6 & 25 & 0.016 &  &  &  &  &  &  \\ \bottomrule
          \end{tabular}
       \end{table}      
      
  \section{Ergebnisse und Diskussion}
    
    Mithilfe der Daten in Tabelle \ref{tab:Messdaten} errechnen sich folgende Werte: $K_{L,c} = $ \num[separate-uncertainty]{2.90 \pm 1 e-5} ($s = ,\alpha = 0.95, N = 3$), $pK_{L,c} = $ \num[separate-uncertainty]{3.75 \pm 0.3} ($s = ,\alpha = 0.95, N = 3$), $L_{M} = $\SI[mode=text, separate-uncertainty]{3.75 \pm 0.3}{\mole\per\liter} ($s = ,\alpha = 0.95, N = 3$) $\equiv$ \SI[mode=text, separate-uncertainty]{3.75 \pm 0.3}{\gram\per\liter} ($s = ,\alpha = 0.95, N = 3$).
    
    Eine Auftragung der $pK_{L,c}$ Werte von Tabelle \ref{tab:Messdaten} gegen $[\ch{Pb\pch[2]\aq}]_{eq.}$ ergibt Diagramm.
    
  \pagebreak
  
  \listofreactions
  \printbibliography[title=Literaturverzeichnis]
  \listoffigures
  \listoftables
  
\end{document}
