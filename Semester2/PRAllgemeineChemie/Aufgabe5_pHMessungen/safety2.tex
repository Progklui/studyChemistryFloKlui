\documentclass{article}
\usepackage[utf8]{inputenc}
\usepackage[english,ngerman]{babel}

%% ========================================================================
%%%% MISC usepackages
%% ========================================================================

%% Chemistry
\usepackage{chemfig,chemmacros}
\chemsetup{modules = all}
\chemsetup[redox]{explicit-sign = true}
\chemsetup[phases]{pos=sub}
%\chemsetup[reactions]{before-tag = {R}, tag-open = [, tag-close = ]}
  
%% Maths
\usepackage{amsmath,amssymb,amsthm,textcomp}

%% Physics
\usepackage{siunitx}

%% Graphics
\usepackage{graphicx}
\usepackage{tikz}
\usepackage{rotating}
%\usepackage{subfig}

%% Tables and Lists
\usepackage{enumerate}
\usepackage{multicol}
\usepackage{geometry}
\usepackage{tabu}
\usepackage{listings}
\usepackage{tabularx}

%% Structures and Style
\usepackage{caption}
\usepackage{subcaption}
\usepackage{booktabs}
\usepackage{colortbl}

\usepackage{xcolor}
\usepackage{xfrac}
\usepackage[export]{adjustbox}[2011/08/13]

\usepackage{booktabs}
\usepackage{float}

\usepackage{fancyhdr}

%% Citing and Settings
\usepackage[backend=biber,
style=numeric,
backref=true, 
natbib=true, %% offering natbib-compatible commands
hyperref=true, %% using hyperref-package references
sorting= none,
doi=true,
maxcitenames=10,
maxbibnames=100,
citestyle=numeric
]{biblatex}

\addbibresource{references.bib}

\usepackage[toc,automake]{glossaries}
\include{abbrevations}
\makeglossaries

\usepackage[colorlinks=true,linkcolor=blue]{hyperref}

%% Figure settings
\renewcommand{\figurename}{Abbildung}
\renewcommand{\tablename}{Tabelle}
\renewcommand{\listfigurename}{Abbildungsverzeichnis}
\renewcommand{\listtablename}{Tabellenverzeichnis}

%% ========================================================================
%%%% Document Information
%% ========================================================================

%% Title
\title{\pH-Experimente \cite{Versuchsvorschrift}} % Title
\author{Autor: Florian \textsc{Kluibenschedl}} % Author name
\date{Bericht verfasst am: \today} % Date for the report

% Page style - headers
\pagestyle{fancy}
\fancyhf{}
\rhead{PR Allgemeine Chemie A - SS2019}
\lhead{Institut für Allgemeine Chemie - Universität Innsbruck}
\rfoot{Experiment 5 - Seite \thepage}

\begin{document}
  \renewtagform{reaction}[Rgl. ]{}{}
  
  \maketitle % Insert the title, author and date
  
  \begin{center}
    \begin{tabular}{r p{4cm}}
      Versuchsdurchführung am: & 04. März 2019\\ % Date the experiment was performed
      Gruppe, Matrikelnummer: & 3, 11805747 \\
      Lehrveranstaltung: & PR Allgemeine Chemie A \\
      Institut: & Allgemeine, Anorganische und Theoretische Chemie \\
      Assistent: & Fuhrmann Gerda % Instructor/supervisor
    \end{tabular}
  \end{center}


  \begin{abstract}
    
  \end{abstract}
  
  \pagebreak
  
  \section{Theoretische Grundlagen}
  
    \subsection{Motivation} \label{sec:Motivation}
      Säuren und Basen spielen bei der Betrachtung vieler chemischer Prozesse eine bedeutende Rolle. Wesentliche Begriffe sind dabei die Säure- bzw. Basenstärke (ausgedrückt durch \pKa und \pKb) und der \pH- bzw. \pOH-Wert. Um die Wirkungsweise von Säuren und Basen besser zu verstehen, ist eine Kenntniss dieser Größen unumgänglich.  
  
    \subsection{Ziel des Experiments}
    
      Auf Basis der obigen Überlegungen ist das Ziel, die Eigenschaften dreier Säure-Base Systeme näher zu untersuchen. Dazu soll zunächst der \pKa von Essigsäure bestimmt werden. Mit diesem Ergebnis wird der \pH-Wert einer Natriumacetat Lösung bestimmt und abschließend wird ein Essigsäure-Natriumacetat Puffer näher untersucht.
    
  \section{Experimenteller Teil}
    
    \subsection{Verwendete Materialien}
              
      \begin{table}[H]
        \centering
        \caption[Materialienliste, Quelle: Autor]{Auflistung der verwendeten Geräte und Chemikalien}
        \label{tab:Materialien}
        
        \begin{tabular}{@{}ll|ll@{}}
          \toprule
            Geräte & Hersteller & Chemikalie & Hersteller \\ \midrule
            pH Meter &  & Standard-Pufferlösung - $\pH=7$ &  \\
            pH Messkette &  & Standard-Pufferlösung - $\pH=4.01$ &  \\
            \SI[mode=text,separate-uncertainty]{25.000(45)}{\milli\litre} Vollpipette &  & \SI[mode=text,separate-uncertainty]{0.100}{M} Essigsäure &  \\
            \SI[mode=text,separate-uncertainty]{10.0(1)}{\milli\litre} Bürette &  & \SI[mode=text,separate-uncertainty]{0.100}{M} Natriumacetat &  \\
            \SI[mode=text,separate-uncertainty]{50}{\milli\litre} Bechergläser &  & \SI[mode=text,separate-uncertainty]{0.100}{M} \ch{NaOH} &  \\
            Magnetrührer &  & \SI[mode=text,separate-uncertainty]{0.100}{M} \ch{HCl} &  \\
            Rührfische &  & deionisiertes Wasser &  \\ \bottomrule
        \end{tabular}
      \end{table}
    
    \subsection{Kalibrierung des pH-Messgerätes}  \label{sec:Kalibrierung}
      
      Die beiden Pufferlösungen mit bekanntem \pH-Wert wurden in je ein \SI[mode=text]{50}{\milli\litre} Becherglas gefüllt. Ein Rührfischchen wurde hinzugegeben und das Becherglas auf einen Magnetrührer gestellt. Während den Messungen wurde langsam gerührt ($\SI[mode=text]{60}{rpm}$), um eine möglichst homogene Verteilung der Ionen zu ermöglichen. Das \pH-Meter wurde mit den Pufferlösungen bei $\pH = 7, T = \SI[mode=text]{22.5}{\degreeCelsius}$ und $\pH = 4.01, T = \SI[mode=text]{22.6}{\degreeCelsius}$ kalibriert\footnote{die Handhabung des \pH-Meter erfolgte dabei wie in der Versuchsvorschrift beschrieben und wird deswegen nicht extra angeführt}. Die gerätinterne Kalibrierfunktion berechnete die Werte für die Steigung ($k = \SI[mode=text]{-58.4}{\milli\volt\per\pH}$) und den Ordinatenabschnitt ($d = \SI[mode=text]{1.6}{\milli\volt}$) der Kalibriergeraden. Dies ergibt folgende Kalibriergerade:
      
      \begin{equation}
        \Delta E = k * \pH + d = -58.4 * \pH + 1.6 
      \end{equation}
      
    \subsection{Beschreibung der Versuchsdurchführungen} \label{sec:Versuch}
    
      Die jeweilige Probelösung wurde in einem \SI[mode=text]{50}{\milli\litre} Becherglas vorgelegt, ein Rührfisch hinzugegeben und das Becherglas auf einen Magnetrührer gestellt. Während den Messungen wurde langsam gerührt (ca. $\SI[mode=text]{60}{rpm}$). Der \pH-Wert wurde bei jedem Versuch 3 mal gemessen. Zwischen den Messungen wurde die Messkette mit deionisiertem Wasser gespült und einem Papiertuch abgetrocknet, um unerwünschten Konzentrationsveränderungen durch Fremdionen bzw. zusätzliche Flüssigkeit vorzubeugen. 
      
      \begin{table}[H]
        \centering
        \caption[Probelösungen für die pH Messungen, Quelle: Autor]{Auflistung der verwendeten Probelösungen für die pH Messungen}
        \label{tab:ProbeL}
        
        \begin{tabular}{@{}l|p{12cm}@{}}
          \toprule
            Experiment & Zusammensetzung der Probelösungen inkl. Mengenangabe \\ \midrule \midrule
            siehe \ref{sec:pKAEssigs} & ca. \SI[mode=text]{40}{\milli\liter} \SI[mode=text]{0.100}{M} Essigsäure \\ \midrule
            siehe \ref{sec:HydrolyseNatrium} & ca. \SI[mode=text]{40}{\milli\liter} \SI[mode=text]{0.100}{M} Natriumacetat \\
             & ca. \SI[mode=text]{40}{\milli\liter} Leitungswasser aus dem Hahn \\ 
             & ca. \SI[mode=text]{40}{\milli\liter} deionisiertes Wasser \\ \midrule
            siehe \ref{sec:SauerBase} & (I): ca. \SI[mode=text]{40}{\milli\liter} einer Lösung von je \SI[mode=text]{25.0}{\milli\liter} (Vollpipette) \SI[mode=text]{0.100}{M} Essigsäure und \SI[mode=text]{0.100}{M} Natriumacetat \\
             & (II): zu (I) wurde \SI[mode=text]{1}{\milli\liter} \SI[mode=text]{0.100}{M} \ch{HCl} zugegeben (aus einer Bürette, der Flüssigkeitsstand im Becherglas vor und nach der Zugabe wurde notiert) - pH und Temperaturmessung erfolgten erst nach starkem Rühren (für \SI[mode=text]{1}{\minute}) \\
             & (III): zu (II) wurden \SI[mode=text]{2}{\milli\liter} \SI[mode=text]{0.100}{M} \ch{NaOH} zugegeben (aus einer Bürette, der Flüssigkeitsstand im Becherglas vor und nach der Zugabe wurde notiert) - pH und Temperaturmessung erfolgten erst nach starkem Rühren (für \SI[mode=text]{1}{\minute}) \\ \bottomrule
        \end{tabular}
      \end{table}
    
    \pagebreak
    
    \subsection{Bestimmung des \pKa von Essigsäure} \label{sec:pKAEssigs}
      
      In \ref{rec:EssigsProtolyse} ist die Protolysereaktion der Essigsäure dargestellt. Wie man sieht, erhöht sie die Konzentration der Hydronium-Ionen, weswegen ein \pH-Wert im sauren Bereich erwartet wird. Die Messung des \pH-Wertes erfolgte wie in \ref{sec:Versuch} beschrieben.\\
      
      \begin{reaction}
        CH3COOH + H2O <<=> CH3COO\mch[] + H3O\pch \label{rec:EssigsProtolyse} \\
      \end{reaction} \\
      
      Die Säurekonstante \Ka berechnet sich wie in \eqref{eq:Ka} beschrieben. $c_{0}$ ist dabei die Anfangskonzentration der Essigsäure, x die Konzentration von \ch{H3O\pch} und \ch{CH3COO\mch}. Um von \Ka anschließend den \pKa berechnen zu können, wird eine dimensionslose Größe benötigt\footnote{der Logarithmus ist nur für dimensionslose Größen definiert}, weswegen die Konzentration jeder Spezies durch die Standardkonzentration ($c_{Standard} = \SI[mode=text]{1}{M}$) dividiert wird\footnote{diese Überlegung gilt auch für alle anderen analogen Überlegungen im weiteren Verlauf, weswegen dies nicht mehr erwähnt wird}. Des weiteren wird angenommen, dass die Hydronium-Ionen Konzentration der Autoprotolyse vernachlässigt werden kann. ($10^{-7} << 10^{-3}$. \\
      
      \begin{equation}
        \Ka = \frac{\frac{[\ch{CH3COO\mch\aq}]_{eq.}}{c_{Standard}}*\frac{[\ch{H3O\pch\aq}]_{eq.}}{c_{Standard}}}{\frac{[\ch{CH3COOH\aq}]_{eq.}}{c_{Standard}}} = \frac{[\ch{CH3COO\mch\aq}]_{eq.}^2}{[\ch{CH3COOH\aq}]_{eq.}} = \frac{x^2}{c_{0}-x} \label{eq:Ka}
      \end{equation} \\
      
      Wird angenommen, dass Essigsäure eine schwache Säure ist, kann $x$ gegenüber $c_{0}$ ($c_{0} >> x$) vernachlässigt werden und es ergibt sich \eqref{eq:Kazwei}.
      
      \begin{equation}
        \Ka \approx \frac{x^2}{c_{0}} \label{eq:Kazwei}
      \end{equation}
      
      Der Protolysegrad kann mit \eqref{eq:Protolysegrad} berechnet werden.
      
      \begin{equation}
        \alpha = \frac{[\ch{CH3COO\mch}]_{eq.}}{[\ch{CH3COOH}]_{0}} = \frac{x}{c_{0}} \label{eq:Protolysegrad}
      \end{equation}
      
      \subsubsection{Messergebnisse und Literaturwerte} \label{sec:MessergebnisseEssigs}
      
        \begin{table}[H]
          \centering
          \caption[Messdaten von \ref{sec:pKAEssigs} und daraus abgeleitete Größen, Quelle: Autor]{Messdaten und daraus abgeleitete Größen}
          \label{tab:MessdatenEssigs}
            \begin{tabular}{@{}ll|llll@{}}
              \toprule
               T in \si{\degreeCelsius} & \pH-Wert & [\ch{H3O\pch}] in M & \Ka & $K_{s,circa}$ & $\alpha$ \\ \midrule
               23.8 & 2.86 & \num{1.38e-3} & \num{1.93e-5} & \num{1.90e-5} & 0.0138 \\
               24.1 & 2.86 & \num{1.38e-3} & \num{1.93e-5} & \num{1.90e-5} & 0.0138 \\ 
               24.1 & 2.87 & \num{1.35e-3} & \num{1.85e-5} & \num{1.82e-5} & 0.0135 \\ \bottomrule
            \end{tabular}
         \end{table}
       
       \subsubsection{Ergebnisse und Diskussion} \label{sec:ErgebnisseEssig}
       
         Aus den oben angegebenen Daten errechnen sich folgende Werte: 
         
         \begin{center}
           $\Ka = \num[separate-uncertainty]{1.90 \pm 0.09 e-5}$ ($s = \num{\pm 4.62e-7},\alpha = 0.05, N = 3$) \\
           
           $\pKa = \num[separate-uncertainty]{4.72 \pm 0.02}$ ($s = \num{\pm 0.012},\alpha = 0.05, N = 3$) \\
           
           $K_{s,circa} = \num[separate-uncertainty]{1.87 \pm 0.09 e-5}$ ($s = \num{\pm 4.62e-7},\alpha = 0.05, N = 3$) \\
           
           $\pKa _{circa} = \num[separate-uncertainty]{4.73 \pm 0.02}$ ($s = \num{\pm 0.012},\alpha = 0.05, N = 3$) \\
           
           $\alpha = \num[mode=text,separate-uncertainty]{0.0137 \pm 0.0003}$ ($s = \num{\pm 1.73e-4},\alpha = 0.05, N = 3$) \\
         \end{center}
         
         Die gesuchten Verhältnisse der jeweiligen Konstanten: $\frac{K_{s,circa}}{\Ka} = 0.984$, $\frac{\pKa _{circa}}{\pKa} = 1.002$. Aufgrund dieser geringen Abweichungen (unter \SI[mode=text]{2}{\percent}) eignet sich die näherungsweise Berechnung, was bedeutet, dass die Vernachlässigung von x im Nenner von \eqref{eq:Ka} ($x << c_{0}$) zulässig ist. \\
         
         Der etwas geringere Wert im Vergleich zum Literaturwert ($\pKa = 4.76$ bei $T = \SI[mode=text]{25}{\degreeCelsius}$) ist äquivalent zu einem größeren $K_{s}$. Dass die Säure stärker dissoziiert ist, kann aufgrund der geringeren Temperatur ($T \approx \SI[mode=text]{24.0}{\degreeCelsius}$) nicht der Fall sein. Eine Begründung könnte das Lösen von \ch{CO2\gas} im Wasser sein, das durch Bildung von Kohlensäure den \pH-Wert senkt und $[\ch{H3O\pch}]$ - also $K_{s}$ erhöht. Unter Berücksichtigung dieser Effekte wird die Messung als erfolgreich angesehen. Auch die Annahme, dass für die Berechnungen die \ch{H3O\pch} Konzentration der Autoprotolyse von Wasser vernachlässigt werden kann, wurde bestätigt ($10^{-7} << 10^{-3}$). \\
         
         Um die theoretische Konzentration der Essigsäure zu berechnen, wird der Mittelwert der in Tabelle \ref{tab:MessdatenEssigs} angegebenen Konzentrationen der Hydronium-Ionen ($= \SI[mode=text]{1.87e-3}{M}$) und der errechnete \Ka in \eqref{eq:Ka} eingesetzt und auf $c_{0}$ umgeformt. Man erhält als theoretische Konzentration der Essigsäure $c_{0} = \SI[mode=text]{0.100}{M}$. Dies entspricht genau der eingesetzten Konzentration der Essigsäure, was aber wenig verwunderlich ist, da diese ja verwendet wurde, um $K_{a}$ zu berechnen\footnote{sofern die einzelnen Messerte nicht extrem stark voneinander abweichen kann dies als Rückeinsetzen in die Gleichung gesehen werden, was notgedrungen das erwünschte Ergebnis erzielt!}. \\
         
         Um \SI[mode=text]{500}{\milli\liter} einer \SI[mode=text]{0.100}{M} Essigsäure-Lösung herzustellen, werden \SI[mode=text]{4.72}{\milli\liter} einer \SI[mode=text]{60}{\percent}-igen Essigsäure ($\rho = $ \SI[mode=text]{1.06}{\gram\per\milli\liter}, $M_{\ch{CH3COOH}} = $ \SI[mode=text]{60.05}{\g\per\mole}) benötigt. Die Berechnung erfolgte dabei wie \eqref{eq:Konzentration} und \eqref{eq:Volumen} in beschrieben.
       
         \begin{equation}
           c_{60\%} = 0.6 * \frac{\rho}{M_{\ch{CH3COOH}}}  \label{eq:Konzentration}
         \end{equation}
       
         \begin{equation}
           V_{60\%} = V_{0.1 M} * \frac{c_{0.1 M}}{c_{60\%}} = \frac{5}{3} * V_{0.1 M} * \frac{c_{0.1 M} * M_{\ch{CH3COOH}}}{\rho} \label{eq:Volumen}
         \end{equation}
       
         Zur Herstellung der Lösung eignet sich z. B. eine geeichte Messpipette oder eine Bürette, mit der \SI[mode=text]{4.72}{\milli\liter} der \SI[mode=text]{60}{\percent}-igen Essigsäure quantitativ in einen \SI[mode=text]{500}{\milli\liter} Messkolben\footnote{der auf dieses Volumen geeicht ist} transferiert werden.
          
    \pagebreak
    
    \subsection{Hydrolyse von Natriumacetat} \label{sec:HydrolyseNatrium}
     
       Natriumacetat ist die konjugierte Base von Essigsäure und reagiert mit \ch{H2O} wie in \ref{rec:NaAcProtolyse} beschrieben. Die Messung des \pH-Wertes erfolgte wie in \ref{sec:Versuch} beschrieben.
       
       \begin{reaction}
         CH3COO\mch\aq{} + H2O <=>> CH3COOH\aq{} + OH\mch\aq \label{rec:NaAcProtolyse} \\
       \end{reaction} \\
       
       Die entsprechende Basenkonstante \Kb berechnet sich wie in \eqref{eq:Kb} beschrieben ($c_{0} = \SI[mode=text]{0.100}{\mole\per\liter}$). Der Wert von $K_{s}$ wurde in \ref{sec:MessergebnisseEssigs} berechnet. Durch Auflösen der quadratischen Gleichung nach $x$ kann $[\ch{OH\mch\aq}]$ berechnet werden. Daraus errechnet sich $[\ch{H3O\pch\aq}]$ und der gesuchte theoretische \pH-Wert. 
       
       \begin{equation}
         \Kb = \frac{[\ch{CH3COOH\aq}]_{eq.} * [\ch{OH\mch\aq}]_{eq.}}{[\ch{CH3COO\mch\aq}]_{eq.}} = \frac{x^2}{c_{0}-x} = \frac{\num{e-14}}{\Ka} \label{eq:Kb}
       \end{equation}
       
       Wird die \pH-Wert Berechnung nur näherungsweise durchgeführt, wird \eqref{eq:Kbzwei} verwendet.
       
       \begin{equation}
         \Kb \approx \frac{x^2}{c_{0}} \label{eq:Kbzwei}
       \end{equation}
       
       Man erhält demnach folgende theoretische Werte für den \pH (\SI[mode=text]{0.100}{M} \ch{NaCH3COO}): $\pH_{theor.} =  \num[separate-uncertainty]{8.86 \pm 0.1}$ ($s = \num{\pm 0.0053},\alpha = 0.05, N = 3$), $\pH_{circa} = \num[separate-uncertainty]{8.86 \pm 0.1}$ ($s = \num{\pm 0.0053},\alpha = 0.95, N = 3$). Die Näherung erweist sich also auch hier als annehmbar. \\
       
       Der \pH-Wert von Leitungswasser wurde \SI[mode=text]{25}{\second} und \SI[mode=text]{2}{\minute} nach Probenahme bestimmt. Bei den erneuten Messungen wurden immer neue Proben genommen. Die Messung des \pH-Wertes erfolgte wie in \ref{sec:Versuch} beschrieben. \\
       
       Der \pH-Wert von deionisiertem Wasser wurde \SI[mode=text]{15}{\second} und \SI[mode=text]{2}{\minute} nach Probenahme bestimmt. Bei den erneuten Messungen wurden immer neue Proben genommen. Die Messung des \pH-Wertes erfolgte wie in \ref{sec:Versuch} beschrieben.
       
       \subsubsection{Messergebnisse und Literaturwerte}
       
         \begin{table}[H]
          \centering
          \caption[Messdaten der \ch{NaCH3COO\aq}-Lösung, Quelle: Autor]{Messdaten der \ch{NaCH3COO\aq}-Lösung}
          \label{tab:MessdatenNatriumAc}
            \begin{tabular}{@{}ll|l@{}}
              \toprule
               T in \si{\degreeCelsius} & \pH-Wert & [\ch{H3O\pch}] in M \\ \midrule
               21.8 & 7.45 & \num{3.55e-8} \\
               22.0 & 7.46 & \num{3.47e-8} \\ 
               21.9 & 7.50 & \num{3.16e-8} \\ \bottomrule
            \end{tabular}
         \end{table}
         
         \begin{table}[H]
          \centering
          \caption[Messdaten von Leitungswasser, Quelle: Autor]{Messdaten von Leitungswasser}
          \label{tab:MessdatenNatriumAcLeitungs}
            \begin{tabular}{@{}lll|lll@{}}
              \toprule
               $t = \SI[mode=text]{25}{\second}$: & T in \si{\degreeCelsius} & \pH-Wert & $t = \SI[mode=text]{2}{\minute}$: & T in \si{\degreeCelsius} & \pH-Wert \\ \midrule
                 & 20.6 & 7.16 &  & 21.5 & 7.57 \\
                 & 20.7 & 7.89 &  & 21.5 & 8.04 \\ 
                 & 20.5 & 7.61 &  & 21.5 & 7.97 \\ \bottomrule
            \end{tabular}
         \end{table}  
     
         \begin{table}[H]
          \centering
          \caption[Messdaten von deionisiertem Wasser, Quelle: Autor]{Messdaten von deionisiertem Wasser}
          \label{tab:MessdatenNatriumAcdeionWasser}
            \begin{tabular}{@{}lll|lll@{}}
              \toprule
               $t = \SI[mode=text]{15}{\second}$: & T in \si{\degreeCelsius} & \pH-Wert & $t = \SI[mode=text]{2}{\minute}$: & T in \si{\degreeCelsius} & \pH-Wert \\ \midrule
                 & 20.7 & 8.48 &  & 21.5 & 7.97 \\
                 & 21.3 & 8.10 &  & 21.6 & 7.59 \\ 
                 & 21.4 & 7.60 &  & 21.9 & 7.15 \\ \bottomrule
            \end{tabular}
         \end{table}
       
       \subsubsection{Ergebnisse und Diskussion}
         
         Der gemessene \pH-Wert der \ch{NaCH3COO} Lösung besitzt folgenden Wert: $\pH_{exp.} = \num[separate-uncertainty]{7.47 \pm 0.05}$ ($s = \pm 0.026,\alpha = 0.05, N = 3$). Um bessere Ergebnisse zu bekommen, ist evtl. eine exaktere Kalibrierung des \pH-Meters im basischen durchzuführen. Die Abweichung von diesem Messwert zum theoretischen berechneten (siehe \ref{sec:HydrolyseNatrium}) lässt sich durch das Lösen von \ch{CO2\gas} aus der Luft erklären. Ob diese Erklärung ausreicht, den doch recht großen Unterschied (mehr wie eine \pH-Einheit) zu erklären. \\
         
         Der gemessene \pH-Wert des Leitungswassers besitzt folgenden Wert: $\pH = \num[separate-uncertainty]{7.55 \pm 0.70}$ ($s = \pm 0.37,\alpha = 0.05, N = 3$) nach \SI[mode=text]{25}{\second} und $\pH = \num[separate-uncertainty]{7.86 \pm 0.50}$ ($s = \pm 0.25,\alpha = 0.05, N = 3$) nach \SI[mode=text]{2}{\minute}. Damit liegt der \pH-Wert etwas im alkalischen Bereich, jedoch immer noch in dem für Trinkwasser zu empfehlenden Bereich zwischen 6 und 9 \cite{LeitungswasserRichtlinien}. Die Basizität kann durch erhöhten Carbonat-Gehalt erklärt werden\footnote{das CCB bezieht demnach das Leitungswasser vom Kalkgebirge}. \ch{CO3\mch[2]} reagiert in wässriger Lösung bekanntlich als Base. Für die beobachtete systematische Erhöhung des \pH-Wertes nach \SI[mode=text]{2}{\minute} konnte keine sinnvolle Erklärung gefunden werden. Das Lösen von \ch{CO2} aus der Luft würde den \pH-Wert ja senken. Vielleicht kommt es durch das Rühren mit zunehmender Zeit zu einer besseren Verteilung von \ch{CO3\mch[2]}, sodass diese \textit{wirksamer} als Base reagieren können - zugegeben eine wenig fundierte Theorie. Wahrscheinlicher ist ein systematischer Fehler. Die größeren Vertrauensbereiche im Vergleich zu den Messungen für Essigsäure - \ref{sec:ErgebnisseEssig}, Natriumacetat und das Puffersystem - \ref{sec:PufferErgebnisse} bedarfen auch einer Erklärung. \\
         
         Der gemessene \pH-Wert des deionisierten Wasser besitzt folgenden Wert: $\pH = \num[separate-uncertainty]{8.06 \pm 0.81}$ ($s = \pm 0.44,\alpha = 0.05, N = 3$) nach \SI[mode=text]{15}{\second} und $\pH = \num[separate-uncertainty]{7.57 \pm 0.75}$ ($s = \pm 0.41,\alpha = 0.05, N = 3$) nach \SI[mode=text]{2}{\minute}. Deionisiertes Wasser besitzt keine gelösten Salz-Ionen. Die einzigen Ionen, die einen Stromfluss ermöglichen können, sind die durch Autoprotolyse entstehenden Hydronium- und Oxonium-Ionen, wobei deren Konzentration relativ gering ist (jeweils \SI[mode=text]{d-7}{M}). Deionisiertes Wasser ist demnach praktisch ein Nichtleiter, was die \pH-Messung mit einer \pH-Messkette, bei der Strom fießen muss, erschwert. Dies erklärt auch die großen Vertrauensbereiche der Messung. Die beobachtete systematische Senkung des \pH-Wertes kann durch das Lösen von \ch{CO2\gas} aus der Luft erklärt werden. \\
         
         Um \SI[mode=text]{250}{\milli\liter} einer \SI[mode=text]{0.1}{M} \ch{NaCH3COO}\footnote{$M_{\ch{NaCH3COO}} = \SI[mode=text]{136.08}{\gram\per\mole}$} Lösung herzustellen, werden \SI[mode=text]{3.4}{\gram} benötigt. Dazu wird die entsprechende Menge auf einem Wägepapier abgewogen und quantitativ in einen \SI[mode=text]{250}{\milli\liter} Maßkolben transferiert. Dieser wird unter homogenisieren bis zur Marke aufgefüllt. 
    
    \pagebreak
    
    \subsection{Säure/Base Puffer} \label{sec:SauerBase}
    
      Puffersysteme besitzen die Eigenschaft, den \pH-Wert in einem gewissen Bereich konstant zu halten. Das bedeutet, dass sich der \pH-Wert nach Zugabe einer starken Säure bzw. Base nicht erwartungsgemäß stark ändert. Puffersysteme bestehen aus einem konjugierten Säure/Base System einer schwachen Säure mit der konjugierten, starken Base bzw. einer starken Säure mit der konjugierten, schwachen Base. Wie in \ref{sec:pKAEssigs} bestimmt, handelt es sich bei der Essigsäure um eine schwache Säure, mit der zugehörigen starken Base \ch{CH3COO\mch}. Sie kann somit ein Puffersystem bilden. Im Folgenden soll der \pH-Wert eines \ch{CH3COOH}/\ch{CH3COO\mch} Puffers im Verhältnis 1:1 und die \pH-Änderung nach Zugabe von \ch{HCl} bzw. \ch{NaOH} bestimmt werden. Die zugehörigen Reaktionsgleichungen lauten wie folgt:
      
      \begin{reactions}
        CH3COO\mch\aq{} + HCl\aq{} &-> CH3COOH\aq{} + Cl\mch\aq{} \label{rec:Protonenzugabe} \\
        CH3COOH\aq{} + OH\mch\aq{} &-> CH3COO\mch\aq{} + H2O 
      \end{reactions} 
      
      Zur Berechnung des \pH-Werts einer Pufferlösung vor und nach der Zugabe der Säure bzw. Base verwendet man die Henderson-Hasselbach Gleichung:
      
      \begin{equation}
        \pH = \pKa + \log _{10} \frac{[\ch{CH3COO\mch\aq}]}{[\ch{CH3COOH\aq}]} \label{eq:HendersonHasselbach}
      \end{equation}
      
      Es wird erwartet, dass der \pH-Wert des \ch{CH3COOH}/\ch{CH3COO\mch} (1:1) Puffers dem \pKa Wert der Essigsäure entspricht (in \ref{sec:pKAEssigs} bestimmt). Setzt man in \eqref{eq:HendersonHasselbach} die beiden Konzentrationen ein, ergibt sich aufgrund dem 1:1 Verhältnis der Konzentrationen $\log _{10} 1 = 0$ und damit $\pH = \pKa = \num[separate-uncertainty]{4.72 \pm 0.02}$ ($s = \num{\pm 0.012},\alpha = 0.05, N = 3$). \\
      
      Wird \SI[mode=text]{1}{\milli\liter} einer \SI[mode=text]{0.1}{M} \ch{HCl} hinzugegeben, errechnet sich der \pH-Wert wie in \eqref{eq:HendersonHasselbachsauer} angegeben (\pKa der Essigsäure in \ref{sec:pKAEssigs} bestimmt). Der Einfachheit wegen wurde mit Stoffmengen anstelle von Konzentrationen gerechnet, da sich die Volumina im Bruch sowieso kürzen.
      
      \begin{equation}
        \pH = \pKa + \log _{10} \frac{n_{\ch{CH3COO\mch\aq}} - 0.0001}{n_{\ch{CH3COOH\aq}} + 0.0001} = \pKa + \log _{10} \frac{0.0025 - 0.0001}{0.0025 + 0.0001} = 4.69 \label{eq:HendersonHasselbachsauer}
      \end{equation}
      
      Werden \SI[mode=text]{2}{\milli\liter} einer \SI[mode=text]{0.1}{M} \ch{NaOH} hinzugegeben, errechnet sich der \pH-Wert wie in \eqref{eq:HendersonHasselbachbasisch} angegeben (\pKa der Essigsäure in \ref{sec:pKAEssigs} bestimmt). Der Einfachheit wegen wurde mit Stoffmengen anstelle von Konzentrationen gerechnet, da sich die Volumina im Bruch sowieso kürzen.
      
      \begin{equation}
        \pH = \pKa + \log _{10} \frac{n_{\ch{CH3COO\mch\aq}} - 0.0001}{n_{\ch{CH3COOH\aq}} + 0.0001} = \pKa + \log _{10} \frac{0.0024 + 0.0002}{0.0026 - 0.0002} = 4.75 \label{eq:HendersonHasselbachbasisch}
      \end{equation}
      
      Ist der \pH-Wert eines \ch{CH3COOH}/\ch{CH3COO\mch} Puffers bekannt, kann mit Gleichung \eqref{eq:Zusammensetzung}, die durch Umstellen von \eqref{eq:HendersonHasselbach} erhalten wurde, dessen Zusammensetzung berechnet werden (\pKa der Essigsäure in \ref{sec:pKAEssigs} bestimmt). Für $\pH=4.5$ ergibt sich $\frac{[\ch{CH3COO\mch\aq}]}{[\ch{CH3COOH\aq}]} = 0.60 = \frac{3}{5}$.
      
      \begin{equation}
        \frac{[\ch{CH3COO\mch\aq}]}{[\ch{CH3COOH\aq}]} = 10^{\pH - \pKa} \label{eq:Zusammensetzung}
      \end{equation}
      
      \subsubsection{Messergebnisse} \label{sec:MessergebnissePuffer}
        
        \begin{table}[H]
          \centering
          \caption[Messdaten des Puffersystems, Quelle: Autor]{Messdaten des Puffersystems}
          \label{tab:MessdatenPuffersystem}
            \begin{tabular}{@{}ll|ll|ll@{}}
              \toprule
               T in \si{\degreeCelsius} & \pH-Wert & T in \si{\degreeCelsius} & \pH-Wert nach \ch{HCl} Zugabe & T in \si{\degreeCelsius} & \pH-Wert nach \ch{NaOH} Zugabe \\ \midrule
                22.2 & 4.69 & 22.7 & 4.64 & 22.6 & 4.74 \\
                22.7 & 4.66 & 22.1 & 4.64 & 22.0 & 4.72 \\ 
                22.5 & 4.70 & 22.3 & 4.66 & 22.7 & 4.75 \\ \bottomrule
            \end{tabular}
         \end{table}
      
      \subsubsection{Ergebnisse und Diskussion} \label{sec:PufferErgebnisse}
      
        Der \pH-Wert des \ch{CH3COOH}/\ch{CH3COO\mch} Puffers beträgt somit: $\pH = \num[separate-uncertainty]{4.68 \pm 0.04}$ ($s = \num{\pm 0.02},\alpha = 0.05, N = 3$). Dieser ist gleich dem \pKa der Essigsäure ($\pKa = \pH$ bei 1:1 Puffer), wie bereits in \ref{sec:SauerBase} erklärt. Dass er etwas geringer wie der in \ref{sec:ErgebnisseEssig} ermittelte ist, erklärt sich dadurch, dass die Lösung durch das Rühren länger an der Luft stehen gelassen wurde, wodurch sich mehr \ch{CO2\gas} lösen konnte. Auch das Einrühren von Luftblasen durch stärkeres Rühren am Anfang begünstigte dies. \\
        
        In Tabelle \ref{tab:Messdatenvergleich} werden die gemessen \pH-Werte, die in \ref{sec:MessergebnissePuffer} präsentiert wurden mit den theoretisch berechneten Werten von \ref{sec:SauerBase} und \ref{sec:weitereBerechnungen} verglichen.
        
        \begin{table}[H]
          \centering
          \caption[Vergleich der Messdaten mit den theoretischen Werten, Quelle: Autor]{Vergleich der Messdaten mit den theoretischen Werten - $\alpha = 0.05, N = 3$}
          \label{tab:Messdatenvergleich}
            \begin{tabular}{@{}l|lllll@{}}
              \toprule
                & \ch{HCl} zu Puffer & \ch{HCl} und \ch{NaOH} zu Puffer & \ch{HCl} zu \ch{H2O} & \ch{NaOH} zu \ch{H2O} \\ \midrule
                gemessen & \num[separate-uncertainty]{4.65 \pm 0.02} ($s = \pm 0.01$) & \num[separate-uncertainty]{4.74 \pm 0.04} ($s = \pm 0.02$) & - & - \\
                berechnet & 4.69 & 4.75 & 2.71 & 11.29 \\ \bottomrule
            \end{tabular}
         \end{table}
         
         Wie zu erwarten konnte beobachtet werden, dass sich der \pH-Wert vom \ch{CH3COOH}/\ch{CH3COO\mch} Puffer nach Zugabe von \ch{HCl} bzw. \ch{NaOH} nicht dramatisch ändert. Eine kleine Änderung ins saure bzw. basische konnte, nicht sonderlich überraschend, beobachtet werden. Die Genauigkeit der Messungen sowie die Übereinstimmung mit den berechneten Werten ist zufriedenstellend. Die Tendenz, dass die gemessenen Werte etwas niedriger im Vergleich zu den Berechneten sind, lässt sich mit dem oben angeführten Argument erklären. 
      
      \pagebreak
      
      \subsubsection{Weitere \pH-Wert Berechnungen} \label{sec:weitereBerechnungen}
      
        Werden zu \SI[mode=text]{50}{\milli\liter} deionisiertem Wasser \SI[mode=text]{1}{\milli\liter} einer \SI[mode=text]{0.1}{M} \ch{HCl} hinzugegeben, errechnet sich der \pH-Wert wie in \eqref{eq:pHWasser} angegeben.
        
        \begin{equation}
          \pH = -\log _{10} \frac{0.001 * 0.1}{0.05 + 0.001} = 2.71 \label{eq:pHWasser}
        \end{equation}\\
        
        Werden zu \SI[mode=text]{50}{\milli\liter} deionisiertem Wasser \SI[mode=text]{1}{\milli\liter} einer \SI[mode=text]{0.1}{M} \ch{HCl} und \SI[mode=text]{2}{\milli\liter} einer \SI[mode=text]{0.1}{M} \ch{NaOH} hinzugegeben, errechnet sich der \pH-Wert wie in \eqref{eq:pHWasserzwei} angegeben. Die Berechnung erfolgt analog dazu, wenn \SI[mode=text]{1}{\milli\liter} einer \SI[mode=text]{0.1}{M} \ch{NaOH} hinzugegeben werden, da der Rest mit der \ch{HCl} neutralisiert wird.\\
        
        \begin{equation}
          \pH = 14 + \log _{10} \frac{0.001 * 0.1}{0.05 + 0.001} = 11.29 \label{eq:pHWasserzwei}
        \end{equation}
        
        Diese dramatischen Änderungen zeigen, dass \ch{H2O}, wie zu erwarten, kein Puffer ist. \\
        
        Die Reaktionsgleichungen eines \ch{NH3}/\ch{NH4Cl} Puffers\footnote{schwache Base mit starker konjugierter Säure} mit \ch{HCl} und \ch{NaOH} lauten wie folgt:
        
        \begin{reactions}
          NH3\aq{} + HCl\aq{} &-> NH4\pch\aq{} + Cl\mch\aq{} \\
          NH4\pch\aq{} + OH\mch\aq{} &-> NH3\aq{} + H2O
        \end{reactions}
        
  \pagebreak
  
  \listofreactions
  \printbibliography[title=Literaturverzeichnis]
  \listoftables
  
\end{document}
